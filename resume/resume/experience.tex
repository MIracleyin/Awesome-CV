%-------------------------------------------------------------------------------
%	SECTION TITLE
%-------------------------------------------------------------------------------
\cvsection{实习经历}


%-------------------------------------------------------------------------------
%	CONTENT
%-------------------------------------------------------------------------------
\begin{cventries}
%---------------------------------------------------------
\cventry
{论文数据标注 \& 训练数据治理 \& 模型评估} % Job title
{IDEA 粤港澳大湾区数字经济研究院(数据算法实习生)} % Organization
{深圳} % Location
{2023 年 4 月 - 至今} % Date(s)
{
  \begin{cvitems}
    \item {\textbf{论文数据标注}。负责数据采集、管理,对接算法工程师对数据的需求,并向数据标注人员传达具体工作内容。阅读文献并根据内容向大语言模型提问,根据文章内容反馈模型输出结果,如输出错误、有害则修改结果,为模型 \textbf{SFT 训练阶段}服务。比较同一个问题的不同模型输出,排序模型输出质量,该数据用于人工评估,后续为模型 \textbf{RLHF 训练阶段服务}。对数据中的重点问题输出进行要点拆分,该数据用于构建模型\textbf{自动化评估}}
    \item {\textbf{模型训练数据构建与治理}。获取 arXiv LaTex 源码数据,对数据进行必要的统计和预处理。统计数据的缺项、非法项,并过滤 LaTeX 源码中不必要的内容。构建 PDF 数据的解析 pipeline,针对不同 PDF 类别制定内容解析策略,构建评估标准。细粒度地解析 PDF 内容中的标题、正文、参考文献等信息,进一步补充训练数据。调研 gpt-neox 的训练数据 the pile 质量,使用 MinHash 等近似字符串匹配方法探索重复出现的字符串,使用单机多卡配置的 \textbf{DeepSpeed} 重新训练 \textbf{gpt-neox 1.3B-20B} 模型后,发现在去重后的数据上训练模型可以降低语言模型的困惑度}
    \item {\textbf{模型能力评估与记忆探针}。在开源的 gpt-neox 上实验探测训练数据提取方法,用作 \textbf{LLaMA} 等大模型的记忆能力探针。在 the pile 数据中构造评估数据,每条数据由 100 个token 构成,向模型提供前 50 个token,设计方法使得模型可以逐字输出后 50 个 token,结果表明模型可以逐字记住训练数据。基于该现象,使用论文摘要、标题数据设计实验,探索大模型在使用这些数据训练后的记忆能力,发现重复训练模型会导致记忆的衰退}
  \end{cvitems}
}

%---------------------------------------------------------
  \cventry
    {算法实现 \& 实验} % Job title
    {Affine 万有理论(算法实习生)} % Organization
    {杭州} % Location
    { 2021 年 12 月 ‑ 2022 年 12 月} % Date(s)
    {
      \begin{cvitems} % Description(s) of tasks/responsibilities
        % \item {构建大规模论文元信息数据库。使用文档间增强的 BERT 模型为论文信息生成表征构建向量数据库。提供 Faiss 查询接口,支持语义相似度搜索}
        \item {\textbf{使用图神经网络增强论文表征}。将论文表征作为节点表征,引用关系作为图输入 GraphSage 网络。在科学文献评估数据集上,大幅提高了引用预测、文章相似度等任务性能}
        \item {\textbf{使用注意力值解释论文相似度}。利用 PyTorch 中的钩子函数,记录 BERT 各层在以相似度为损失函数反向传播时注意力的值,对两篇论文的每个 token 获得可解释性分数。最后使用各类 NER、POS 方法将结果汇聚,以优化可视化效果}
      \end{cvitems}
    }



\end{cventries}
