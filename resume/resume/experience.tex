%-------------------------------------------------------------------------------
%	SECTION TITLE
%-------------------------------------------------------------------------------
\cvsection{研究经历}


%-------------------------------------------------------------------------------
%	CONTENT
%-------------------------------------------------------------------------------
\begin{cventries}

%---------------------------------------------------------
  \cventry
    {算法实现 \& 实验 \& 论文写作} % Job title
    {标签感知推荐系统(研究生毕设课题)} % Organization
    {杭州} % Location
    {2021年7月 - 至今} % Date(s)
    {
      \begin{cvitems} % Description(s) of tasks/responsibilities
        \item {探索现有研究对社会标签数据的建模方式}
        \item {使用轻量化的图神经网络对社会标签数据进行表征学习,相关论文\emph{LFGCF: Light Folksonomy Graph Collaborative Filtering for tag-aware recommendation} 提交至期刊 \emph{Expert Systems With Applications}}
        \item {使用对比学习挖掘社会标签数据,有效的降低了热门物品引入的流行度偏差,相关论文 \emph{A novel graph-based contrastive learning recommender framework for social tagging systems} 已获得期刊 \emph{Information science} 返修意见}
      \end{cvitems}
    }

%---------------------------------------------------------
  \cventry
    {算法实现 \& 实验} % Job title
    {针对科研文档的自然语言处理与分析(实习研究课题)} % Organization
    {杭州} % Location
    {2021年12月 - 2022年12月} % Date(s)
    {
      \begin{cvitems}
        \item {探索科学文档优化的自然语言模型}
        \item {使用图神经网络增强自然语言模型对科学文档的表征性能}
        \item {使用可解释性深度学习技术为相似论文提供词级别的可视化解释}
      \end{cvitems}
    }

\end{cventries}

%-------------------------------------------------------------------------------
%	SECTION TITLE
%-------------------------------------------------------------------------------
\cvsection{实习经历}

%-------------------------------------------------------------------------------
%	CONTENT
%-------------------------------------------------------------------------------
\begin{cventries}
  %---------------------------------------------------------
  \cventry
    {算法实现 \& 实验} % Job title
    {AFFiNE} % Organization
    {杭州} % Location
    {2021年12月 - 2022年12月} % Date(s)
    {
      \begin{cvitems}
        \item {探索科学文档优化的自然语言模型}
        \item {使用图神经网络增强自然语言模型对科学文档的表征性能}
        \item {使用可解释性深度学习技术为相似论文提供词级别的可视化解释}
      \end{cvitems}
    }

\end{cventries}